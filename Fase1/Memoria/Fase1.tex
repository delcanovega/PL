\documentclass[a4paper,11pt]{report}
\usepackage[T1]{fontenc}
\usepackage[utf8]{inputenc}
\usepackage{lmodern}

\title{
  Procesadores de Lenguajes\\
  \large Primera Fase
}

\author{
  Concepción Echeverría, Lidia\\
  \texttt{lidiacon@ucm.es}
  \and
  del Caño Vega, Juan Ramón\\
  \texttt{jdelcano@ucm.es}
}

\begin{document}

\maketitle
\tableofcontents

\chapter*{Clases léxicas}
\addcontentsline{toc}{chapter}{Clases léxicas}

Dispondremos de las siguientes clases léxicas:

\begin{enumerate}
  \item Nombres de tipo: serán palabras reservadas en nuestro analizador léxico.
  \begin{itemize}
    \item $num$: palabra reservada para tipar las variables numéricas.
    \item $bool$: palabra reservada para tipar las variables booleanas.
  \end{itemize}
  \item Nombres de variable: $iden$ serán los identificadores de las variables que declaremos. Deberan empezar por una letra, y pueden contener letras, dígitos o subrayados (\_).
  \item Separadores: dispondremos de dos tipos, uno para separar la seccion de declaraciones de la sección de instrucciones, y otro para separar las declaraciones o instrucciones entre ellas.
  \begin{itemize}
    \item $coma$: será el separador de declaraciones o instrucciones entre ellas mismas (;).
    \item $end$: indicará el final de la sección de declaraciones y el comienzo de la sección de instrucciones (\&\&).
  \end{itemize} 
  \item Expresiones: las usaremos para trabajar en la sección de instrucciones. Tenemos dos tipos.
  \begin{enumerate}
    \item Expresiones lógicas: de nuevo serán palabras reservadas.
    \begin{itemize}
      \item $true$: valor lógico de verdad (true).
      \item $false$: (false) contrario de $true$.
    \end{itemize}
    \item Expresiones numéricas: podrán comenzar por un signo (+ o -), seguido de uno o más dígitos. Si se tratase de números reales, aparecería un punto (.) seguido de uno o más dígitos. Finalmente, para ambos casos, podría aparecer una parte exponencial, con una $e$ o $E$ seguida opcionalmente de un signo (+ o -) y uno o más dígitos. 
  \end{enumerate}
  \item Operadores: para la sección de instrucciones dispondremos de los siguientes operadores.
  \begin{enumerate}
    \item Operador de asignación: $igual (=)$ entre una variable y una expresión.
    \item Operadores aritméticos:
    \begin{itemize}
      \item $mas$: operador de suma (+).
      \item $menos$: operador de resta (-).
      \item $por$: operador de multiplicación (*).
      \item $div$: operador de división (/).
    \end{itemize}
    \item Operadores lógicos: también serán palabras reservadas.
    \begin{itemize}
      \item $and$: operador de conjunción.
      \item $or$: operador de disyunción.
      \item $not$: operador de negación.
    \end{itemize}
    \item Operadores relacionales.
    \begin{itemize}
      \item $mayor$: comprueba si una expresión es estrictamente mayor a otra (>).
      \item $menor$: comprueba si una expresión es estrictamente menor a otra (<).
      \item $mayorIgual$: comprueba si una expresión es mayor o igual a otra (>=).
      \item $menorIgual$: comprueba si una expresión es menor o igual a otra (<=).
      \item $equiv$: comprueba si dos expresiones tienen el mismo valor (==).
      \item $noEquiv$: comprueba si dos expresiones tienen distinto valor (!=).
    \end{itemize}
  \end{enumerate}
\end{enumerate}

\chapter*{Especificación formal}
\addcontentsline{toc}{chapter}{Especificación formal}

\chapter*{Diagrama de transiciones}
\addcontentsline{toc}{chapter}{Diagrama de transiciones}


\end{document}
